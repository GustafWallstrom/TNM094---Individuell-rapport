\documentclass[a4paper,12pt,oneside,final]{extbook}

\usepackage[utf8]{inputenc}
\usepackage[T1]{fontenc}

\usepackage{graphicx}
\usepackage{times}
\usepackage[english,swedish]{babel}

\usepackage{geometry}

\geometry{
 margin=20mm
} 

\usepackage{fancyhdr}

\usepackage{titling}
\title{Individuell rapport, TNM094}
\author{Gustaf Wallström\\Spel på telefon-HMD}

\frenchspacing
\setlength{\parindent}{0pt}
\parskip 5pt

\usepackage{color}
\definecolor{rltred}{rgb}{.5,0,0}
\definecolor{rltgreen}{rgb}{0,.5,0}
\definecolor{rltblue}{rgb}{0,0,1}

\usepackage[pdftex,
 colorlinks=true,
 urlcolor=rltblue,       % \href{...}{...} external (URL)
 filecolor=rltgreen,     % \href{...} local file
 linkcolor=rltred,       % \ref{...} and \pageref{...}
 citecolor=rltgreen,     % \cite{...}
 pdftitle={},
 pdfauthor={},
 pdfsubject={Projektrapport, TNM094},
 pdfkeywords={},
 pdfpagemode=,
 pdfstartview=FitH,
 bookmarks=true,
 bookmarksopen=false,
 bookmarksnumbered=true
        ]{hyperref}


\begin{document}

\pagestyle{empty}
\thispagestyle{empty}

\frontmatter

\maketitle

\pagestyle{fancy}

\chapter{Sammanfattning}

En sammanfattning ska kort och koncist beskriva och motivera det
studerade problemet, metoden samt resultat och slutsatser. Arbetets
bidrag till huvudområdet ska tydligt framgå. Vad är det rapporten
säger om huvudområdet som vi inte visste tidigare?

Sammanfattningens längd växer med längden på rapporten. I en rapport
av denna typ kan den vara tre stycken lång: ett inledande motiverar
arbetet och beskriver bakgrund, ett beskriver redogörelsen och ett
beskriver analys och slutsatser. Sammanfattningen innehåller inga
referenser eller ekvationer.


\tableofcontents

\cleardoublepage
% \phantomsection
\addcontentsline{toc}{chapter}{\listfigurename}
\listoffigures

\cleardoublepage
% \phantomsection
\addcontentsline{toc}{chapter}{\listtablename}
\listoftables

\mainmatter

\chapter{Inledning}
\label{ch:inledning}

Inledningen presenterar rapportens grund för läsaren och ska därför
behandla: motivering, syfte, frågeställning och avgränsning.


\section{Bakgrund}

Vid behov kan man här \emph{kort} presentera vad som motiverade
arbetet. Alternativt läggs det som del av nästa avsnitt, ``syfte''.


\section{Syfte}

Den här rapporten speglar hur ett grupparbete kan planeras, struktureras och utföras. Alla slutsatser och antaganden kommer ursprungligen från 

\section{Frågeställning}

Här ska de specifika frågeställningarna beskrivas. Det ska vara
regelrätta frågor som avslutas med frågetecken. Frågorna ska besvaras
i slutsatser-kapitlet, eller så ska det anges varför de inte kunde
besvaras. Ofta kommer en rapport att innehålla flera olika
frågeställningar som hänger ihop. I vanliga fall brukar det vara två
till fyra frågor.

Exempel på frågeställningar för den individuella rapporten i TNM094
(generaliserade):

\begin{itemize}
\item Kan man använda utnyttja tekniken X i ett projekt om Y för att
  få effekten Z?
\item Hur kan ett system (eller en lösning) för X realiseras så att
  effekten Y uppstår?
\item Vilka alternativ finns för att åstadkomma X och vilket
  alternativ ger bäst effekt avseende Y och Z?  (Denna frågeställning
  bryts lämpligtvis ner i två separata frågor.)
\end{itemize}

Observera att en specifik frågeställning nästan alltid ger ett bättre
arbete än en generell frågeställning (det är helt enkelt mycket
svårare att göra något vettigt av en generell frågeställning). Det är
också fördelaktigt att inte bara skriva frågan, utan förtydliga i ett
par meningar vad som menas med frågan och hur den kan besvaras.

Det bästa sättet att få till en riktigt bra och specifik
frågeställning är att göra en noggrann teorigenomgång och sätta sig in
i relaterad forskning och praktik. Då får man idéer och terminologi på
köpet vilket gör att man kan uttrycka sig precist och även ha något
vettigt att säga i diskussionen. Och har man väl hittat fram till en
detaljerad frågeställning så blir det lättare att arbeta sig fram till
en bra metod och man kan genomföra själva arbetet mycket snabbare än
om man jobbar med vaga formuleringar. Det brukar alltså löna sig i
längden att lägga lite extra tid i början på att göra en ordentlig
teorigenomgång.

Handledaren är behjälplig med att bedöma när frågeställningen är
tillräckligt specifik.


\section{Avgränsningar}

Här beskrivs kortfattat de viktigaste avgränsningar som medvetet
gjorts. Det kan till exempel gälla att man fokuserat arbetet på en
viss tillämpningsdomän eller målgrupp. I normalfallet behöver
avgränsningarna inte motiveras.


\chapter{System och tekniska lösningar}

Här beskrivs systemet som ska byggas i projektet, dess delar och hur
det ska byggas samman. Eftersom den individuella rapporten behandlar
projektet i ett tidigt skede så kommer mycket vara spekulativt, men
bör ändå vara realistiskt och välmotiverat.

För att underlätta lärarent bedömning: Tag bort eller flytta runt
avsnitten inom kapitelet, om men endast om det är nödvändigt för det
specifika projektet, men flytta inte avsnitt mellan kapitel och byt
heller inte titel på avsnitt eller kapitel.

Glöm inte att motivera era val utifrån projektets behov och referera
till eventuella källor.

\section{Grundläggande, initiala krav och systembegränsningar}

Beskrivning av systemets mål och vilka grundläggande krav som sätts
för att detta mål ska anses uppfyllt. Här diskuteras även
grundläggande systembegräsningar såsom hur många användare som ska
kunna nyttja systemet samtidigt, eller hur lång svarstid systemet får
ha för att anses fylla målet.

\section{Målplattform}

Beskrivning av målplattformen och hur t ex tredje-part-programvara
används för att underlätta stödet för eller utvecklingen till
målplattformen.

\section{Grundläggande system-arkitektur}

Beskrivning av systemets grundläggande arkitektur, både interna och
externa komponenter och noder, hur de viktigaste systemdelarna i sig
delas upp i moduler med specifika uppgifter, systemets interna
modulers programspråk, behov av ramverk som stödjer denna
system-arkitektur eller definierar och bygger upp den.

\section{Standarder}

Här beskrivs de standarder som systemet behöver följa, såsom
filformat, protokoll eller andra gränssnitt. Beskriv också vilka
tredjeparts-bibliotek som kommer att användas för att göra detta.

\section{Utvecklings-miljö}

Här beskrivs de verktyg som ska användas i utvecklingen, såsom IDE, kompilator, debug-verktyg, profileringsverktyg, etc.

\chapter{Projekthantering}

Här beskrivs de övergripande teknikerna som övervakar och styr
samarbetet inom projektet.

\section{Utvecklingsmetodik}

Beskriv och motivera här den övergripande strukturen,
utvecklingsmetodiken eller metodikerna, som valts för projektet. Här
ska bara grundläggande principer och motivering diskuteras. Detaljer
diskuteras och motiveras under respektive avsnitt.

\section{Organisation}

Här beskrivs vilka som ingår i projektet, både utvecklare och externa
intressenter, deras ansvarsfördelning (och eventuellt hur den är
planerad att förändras över tid) och eventuella arbetsgrupper och
vilka syften och uppgifter dessa har.

\section{Tidsplan}

Beskriv, diskutera och motivera tidsplan för hela projektet, sprints,
möten och milestones, inklusive tid för planering och leverans.

\section{Milestones och leverabler}

Mer detaljerad beskrivning och motivering av milestones och
leverabler, såsom rapporter, prototyper och färdigt system.

\chapter{Rutiner och principer}

I det här kapitlet beskrivs de rutiner och principer som används av
alla inom projektet för att säkerställa ett fungerande samarbete och
rätt kvalitet på utveckling och slutprodukt.

\section{Mötesprinciper och rutiner}

Beskriv vilka typer av möten gruppen kommer att ha och hur dessa ska
gå till.

\section{Kravhantering och -spårning}

Beskriv hur gruppen arbetar med kravhantering och eventuellt även
kravspårning och vilket teknikstöd som kommer att användas. Med
kravspårning avses processer, rutiner och dokumentation som syftar i att
ge full kontroll över hur varje enskilt krav implementeras i
systemarkitektur, programdesign, programkod, enhetstest,
integrationstest och systemtest.

Här beskrivs också hur projektet säkerställer synkronisering mellan
intressenternas behov och genomförandet.

\section{Versionshantering, -system och rutiner}

Beskriv hur gruppen arbetar med versionshantering och vilket
teknikstöd som kommer att användas.

\section{Arkitektur- och programdesign, standarder och rutiner}

Beskriv hur gruppen arbetar med arkitektur och programdesign.

\section{Dokumentationsprinciper och rutiner}

Beskriv hur gruppen arbetar med dokumentation av programkod men även
möten och andra delar av utvecklingsprocessen och vilket teknikstöd
som kommer att användas.

\section{Kvalitetssäkring}

Beskriv hur gruppen arbetar med kvalitetssäkring av programkod
(granskning och testning) och vilket teknikstöd som kommer att
användas.

\chapter{Analys och diskussion}

Det är här ni analyserar och diskuterar arbetet. I diskussionskapitlet
ska man explicit referera till både andra avsnitt i rapporten och
externa källor som är relevanta för diskussionen.


\section{Resultat}

Finns det något i resultaten som står ut och behöver analyseras och
kommenteras? Vad säger teorin om vad resultaten egentligen betyder?
Finns det något i resultaten som är oväntat baserat på teori och andra
källor, eller stämmer det bra överens med vad man teoretiskt kunde
förvänta sig?

\section{Arbetet i ett vidare sammanhang}

Det ska ingå ett stycke med en diskussion om etiska och samhälleliga
aspekter relaterade till arbetet. Detta är viktigt för att påvisa
professionell mognad samt för att utbildningsmålen ska kunna
uppnås. Om arbetet av någon anledning helt saknar koppling till etiska
eller samhälleliga aspekter ska detta explicit anges i stycket
Avgränsningar i inledningskapitlet.

Exempel på samhälleliga eller etiska aspekter är människors liv och
hälsa, samhällets funktionalitet, demokrati, rättssäkerhet och
mänskliga fri- och rättigheter, miljö och ekonomiska värden samt
nationell suveränitet. Inom planering av/och systemutveckling kan det
exempelvis handla om arbetsbelastning, ekonomisk kompensation,
uppdelning mellan arbetstid och fritid, könsdiskriminering eller annan
form av diskriminering, programvarans användning i terrorism,
kärnvapen, miljökopplad industri eller demokratisk utveckling.


\chapter{Slutsatser}

I detta kapitel ska en återkoppling till syfte och frågeställningar
ske. Har syftet uppnåtts och vad blev svaret på frågeställningarna?
Varje frågeställning kan få ett eget avsnitt för att tydliggöra
strukturen.

Här ska också arbetets konsekvenser för berörd målgrupp och eventuellt
för forskare och praktiker beskrivas. Man kan också ha ett stycke
eller avsnitt om framtida arbete där man beskriver vad man skulle
vilja göra om man hade mer tid eller som rekommendationer för framtida
studier eller exjobb. Om man har ett sådant stycke är det dock viktigt
att det är konkreta och väl genomtänkta förslag som presenteras,
snarare än vaga idéer.


\begin{thebibliography}{99}
\addcontentsline{toc}{chapter}{\bibname}

\bibitem{pfleeger10softwareengineering}
  Shari Lawrence Pfleeger och Joanne M. Atlee, \emph{Software Engineering, Fourth Edition, International Edition}, Pearson 2010

\bibitem{ngnblog} Författare, \emph{Titel}, Organisation, yyyy-mm-dd, hämtad: yyyy-mm-dd\newline http://länk/till/sidan

\end{thebibliography}


\appendix

\chapter{Bilaga}

Efter \textbackslash appendix placeras kapitel som ska behandlas som
appendix. Dessa innehåller delar av rapporten som inte är viktiga för
att kunna ta till sig innehållet, men som ändå är kopplade till
arbetet. Exempel är arbetsfördelningen inom projektet eller site-karta
och skärmdumpar från ett webbsystem som utvecklades inom projektet.


\chapter{Allmänt om språkbruk}

Det finns många guider för språkbruket i rapporter, artiklar och konferensbidrag. Några exempel 
\begin{itemize}
\item Enkla språkregler, Studieverkstan, Högskolan i Skövde, 2014-11-03, hämtad 2014-12-10\newline
  \href{http://www.his.se/Ar-student/Studievagledning/studieverkstan/Enkla-sprakregler/}%
       {http://www.his.se/Ar-student/Studievagledning/studieverkstan/Enkla-sprakregler/}
\item Akademiskt skrivande, Mikael Nygård, Åbo Akademi, hämtad 2014-12-10\newline
  \href{http://www.vasa.abo.fi/users/minygard/Tips2-filer/Akademiskt\%20skrivande2.htm}%
       {http://www.vasa.abo.fi/users/minygard/Tips2-filer/Akademiskt\%20skrivande2.htm}
\item Språkguider, Linköpings universitet\newline
  \href{https://old.liu.se/insidan/kommunikationsstod/sprakguider/sprakguider?l=sv\&sc=true}%
       {https://old.liu.se/insidan/kommunikationsstod/sprakguider/sprakguider?l=sv\&sc=true}
\end{itemize}
För att undvika några vanliga fallgropar:
\begin{itemize}
\item Börja med innehållet, vad är det rapporten handlar om? Tekniker
  såsom \emph{brainstorming} och \emph{mind map} är bra verktyg som är
  väl värda den tid det tar att lära sig.
\item Strukturen är mycket viktig --- läsaren ska handledas genom
  teorier och deras förståelse byggas upp som ett korthus, från grund
  till topp. Genom att börja något under den nivå man tror att läsaren
  ligger på så säkerställer man att läsaren och författarna har en
  gemensam förståelse för vad som presenteras senare.
\item Använd styckesindelning och försök skapa ett flöde genom
  texten. En bra strukturerad och formulerad text går att läsa och
  förstå hur den är strukturerad även om man tar bort kapitel och
  rubriker.
\end{itemize}


\end{document}

